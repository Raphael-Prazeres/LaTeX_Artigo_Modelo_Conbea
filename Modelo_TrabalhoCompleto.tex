\documentclass[a4paper]{article}

% Pacotes adicionais
\usepackage[T1]{fontenc}
\usepackage[utf8]{inputenc}
\usepackage[top=25mm, bottom=25mm, left=25mm, right=25mm]{geometry}
\usepackage{graphicx}
\usepackage{lipsum}
\usepackage{float}

% Configurações de página
\pagestyle{empty}
\setlength{\parindent}{1cm}
\linespread{1}

% Configurações de fonte
\renewcommand{\familydefault}{\rmdefault}
\usepackage{titlesec}
\titleformat*{\section}{\Large\bfseries\uppercase}
\titleformat*{\subsection}{\normalsize\bfseries}
\titleformat*{\subsubsection}{\normalsize\bfseries}
\renewcommand{\thefootnote}{\fnsymbol{footnote}}

% Início do documento
\begin{document}
	
	% Cabeçalho
	\begin{center}
		\includegraphics[width=\textwidth]{cabecalho}
	\end{center}
	
	% Título
	\section*{\centering TÍTULO DO TRABALHO\textsuperscript{1}}
	\textsuperscript{1} Chamada de identificação (se necessário)
	
	% Autores
	\begin{center}
		\large
		Autor 1\textsuperscript{2}, Autor 2\textsuperscript{3}, Autor 3\textsuperscript{4}
	\end{center}
	
	% Identificações dos autores
	\begin{center}
		\footnotesize
		\textsuperscript{2} Engo Agrônomo, Prof. Adjunto, Depto. de Engenharia Rural, Faculdade de Ciências Agronômicas, UNESP, Botucatu - SP, Fone: (0XX14) 3811.7165, dhrip@fca.unesp.br \\
		\textsuperscript{3} Engo Agrícola, Prof. Doutor, Depto. de Engenharia Rural, FCA/UNESP, Botucatu - SP \\
		\textsuperscript{4} Outra identificação do autor 3
	\end{center}
	
	% Identificação do evento
	\begin{center}
		Apresentado no \\
		LII Congresso Brasileiro de Engenharia Agrícola - CONBEA 2023 \\
		18 a 21 de outubro de 2023 – Ribeirão Preto - SP, Brasil
	\end{center}
	
	% Resumo
	\section*{RESUMO}
	O texto do resumo deve ser claro, sucinto e explicar os objetivos, procedimentos, resultados e conclusões do trabalho, com no máximo 14 linhas.
	
	\subsection*{Palavras-Chave}
	Palavra-chave 1; Palavra-chave 2; Palavra-chave 3
	
	% Título em Inglês
	\section*{TÍTULO EM INGLÊS}
	English Title
	
	% Abstract
	\section*{ABSTRACT}
	The abstract text should provide a clear and concise summary of the objectives, procedures, results, and conclusions of the work, within a maximum of 14 lines.
	
	\subsection*{Keywords}
	Keyword 1; Keyword 2; Keyword 3
	
	% Introdução
	\section{INTRODUÇÃO}
	A introdução deve apresentar de forma concisa e clara o problema abordado, a justificativa e os objetivos do trabalho, utilizando no máximo 40 linhas.
	
	% Material e Métodos
	\section{MATERIAL E MÉTODOS}
	Descreva os materiais utilizados, a caracterização da área experimental (se aplicável) e os métodos adotados, referenciando-se apropriadamente quando forem métodos consagrados na literatura.
	
	% Resultados e Discussão
	\section{RESULTADOS E DISCUSSÃO}
	Apresente os resultados e discuta-os, confrontando-os com a literatura. Utilize ilustrações, gráficos e tabelas de forma clara e concisa, numerando-os de acordo.
	
	% Conclusões
	\section{CONCLUSÕES}
	Apresente as conclusões do trabalho com base nos resultados obtidos, evitando repetições e relacionando-os com os objetivos estabelecidos.
	
	% Agradecimentos
	\section*{AGRADECIMENTOS}
	Texto dos agradecimentos (se houver).
	
	% Referências
	\section*{REFERÊNCIAS}
	\begin{enumerate}
		\item Exemplo de referência.
		\item Outro exemplo de referência.
	\end{enumerate}
	
\end{document}
